\documentclass{article}
\usepackage[T1]{fontenc}
\usepackage[utf8]{inputenc}
\usepackage{lmodern}
\usepackage[english]{babel}
\usepackage{math}
\usepackage{unicode}
\usepackage[margin=15mm]{geometry}

\let\fr\mathfrak
\DeclareMathOperator\Cl{Cl}
\def\smat{\small\mat}
\def\F{\mathbb{F}}

\begin{document}
\title{Conjugacy classes of square $2$-adic matrices of size~$2$}

\begin{prop}
There exists group isomorphisms
\begin{align*}
ℤ_2^{×} &≃ (-1)^{ℤ/2ℤ} × (-3)^{ℤ_2};\\
(ℤ_2/2^m ℤ_2)^{×} &≃ \begin{cases} 1, & m = 1;\\
(-1)^{ℤ/2ℤ} × (-3)^{ℤ/2^{m-2}ℤ},& m ≥ 2.\\\end{cases}
\end{align*}
\end{prop}

\begin{prop}
Let~$ℓ$ be a prime number and $f$~be an irreducible polynomial of
degree~two over~$ℤ_{ℓ}$. The set of all conjugacy classes over~$ℤ_{ℓ}$ of
matrices with characteristic polynomial~$f$ is in bijection with the set
of fractional ideal classes of the ring~$ℤ_{ℓ}[α] / (f(α))$.
\end{prop}

\begin{prop}
Let~$K$ be a quadratic extension of~$ℚ_2$.
\begin{enumerate}
\item Let~$θ ∈ K$ such that~$\ro O_K = ℤ_2[θ]$. Then the orders of~$K$
are exactly the rings~$\ro O_n = ℤ_2[2^n θ]$, $n ≥ 0$.
\item The fractional ideal classes of~$\ro O_n$ are the classes of~$\ro
O_m$ for~$0 ≤ m ≤ n$.
\end{enumerate}
\end{prop}

\begin{proof}
A fractional ideal of~$K$ is a $ℤ_2$-lattice~$\fr a ⊂ K$ that is stable
by multiplication by some~$2^n θ$. Moreover, equivalence of fractional
ideals corresponds to equivalence of lattices by multiplication by an
element of~$K^{×}$.

Each lattice is then equivalent to a lattice with basis~$ℤ_2 + τ ℤ_2$
with~$τ ∈ \ro O_K ∖ ℤ_2$. By replacing~$τ$ with~$a τ + b$ with~$a ∈ ℤ_2^{×}$
and~$b ∈ ℤ_2$, we may assume that~$τ = 2^m ω$ for some~$m ≥ 0$,
\emph{i.m.} that $ℤ_2 + τ ℤ_2 = \ro O_m$. The lattice $\ro O_m$~is a $\ro
O_n$-fractional ideal when~$n ≥ m$; conversely, this proves that the
fractional ideals of~$\ro O_n$ are equivalent to the~$\ro O_m$ with~$m ≤
n$.
\end{proof}

\begin{prop}\label{prop:conj-2x2-Z2}
Let~$f(x) = x^2 + px + q$ be a separable polynomial of degree~two
over~$ℤ_2$, and write~$D = p^2 - 4q = d r^2$ with~$r ∈ ℤ_2$ and~$d ∈
\acco {1, -3, -4, 12, ±8, ±24}$. Then there exists exactly~$v_2(r) + 1$
conjugacy classes over~$ℤ_2$ of matrices with characteristic
polynomial~$f$, which are represented by
\begin{equation}\label{eq:conj}
Φ_m = \mat { \frac{-p-rd}{2} & 2^m r \frac{d-d^2}{4} \\
  2^{-m} r & \frac{-p+rd}{2} }, \quad
\text{for $m = 0, …, v_2(r)$.}
\end{equation}
\end{prop}


\begin{proof}
Write~$K = ℚ_2(x)/f(x)$ and~$α ∈ K$ be a root of~$f$.
Let~$θ = \frac{d+√d}{2}$; then~$\ro O_{K} = ℤ_2 ⊕ ℤ_2 θ$. The minimal
polynomial of~$θ$ is~$θ^2 - d θ + \frac{d^2-d}{4} = 0$. Therefore, in the
basis~$(1, 2^m θ)$ of the order~$\ro O_m$, the matrix of~$α = (-p + r
√d)$ takes the form~$Φ_m$ above.
\end{proof}

% \begin{prop}\label{prop:conj-2x2-modulo}
% For any integer~$s ≥ 0$, the conjugacy classes of matrices with
% characteristic polynomial~$f(x) = x^2 + px + q$ over~$ℤ/2^sℤ$ are the
% $\min (s, v_2(r)) + 1$ classes given by the formula~\eqref{eq:conj}
% with~$m ≥ \max (0, v_2(r) - s)$.
% \end{prop}

\begin{prop}
Let~$k$ be a finite field and $E$~be an elliptic
curve over~$k$. For any prime number~$ℓ$ different from the
characteristic of~$k$, the conjugacy class of the Frobenius
endomorphism~$φ_E$ over~$ℤ_{ℓ}$ corresponds to the $ℓ$-adic valuation of
the conductor of the endomorphism ring of~$E$.
\end{prop}

\begin{proof}
By Tate's theorem, the endomorphism ring~$\End_k E$ is isomorphic to the
ring of Galois-equivariant endomorphisms of the Tate module~$T_{ℓ}(E)$.
Let~$\fr a$ be a fractional ideal representing~$φ_E$; then $\End_k E$~is
isomorphic to~$\End_{ℤ_{ℓ}[φ_E]} (\fr a)$. Let~$\ro O$ be the unique
order of~$ℚ_{ℓ}[φ_E]$ equivalent to~$\fr a$; then we see that $\End_k E =
\ro O$.
\end{proof}


Let~$M_c$ be the polynomials for multiplication by a constant~$c$ on~$E$,
in projective coordinates: $c (X:Y:Z) = M_c(X:Y:Z)$.
\begin{prop}
The following algorithm computes the class of the Frobenius endomorphism
of~$E$:
\begin{enumerate}
\def\labelenumi{\arabic{enumi}. }
\itemsep 0pt
\item Compute the characteristic polynomial~$χ$ of~$φ_E$, its
discriminant~$D$, and write~$D = 4^n d$ with $d$~a fundamental
discriminant in~$ℤ_2$.
\item Choose a point~$(α, 0)$ of~$2$-torsion in~$E$ and define the
polynomial~$F_0 = X - α Z$.
\item Let~$λ_0 = 1$ and~$m_0 = M_1 = (X:Y:Z)$.
\item \label{alg:loop1} For all~$i = 0, …, n-1$:
\par\advance \leftskip 3em
\item For all~$t = 0, 1$:
\par\advance \leftskip 3em
\item Let~$λ_{i+1} = λ_i + 2^i t$ and compute the multiplication
polynomials~$m_{i+1} = M_{λ_{i+1}} = (X_{i+1}, Y_{i+1}, Z_{i+1})$.
\item Let~$f_1 = F ∘ μ_2$.
\item Let~$f_2 = X_{i+1} Z^q - Z_{i+1} X^q$.
\item If $f = \mathrm{gcd} (f_1, f2) ≠ 1$, then set~$F = f$ and go back
to step~\ref{alg:loop1}.
\par\advance \leftskip -3em
\item Return $(λ, i)$.
\end{enumerate}
\end{prop}

As examples, we compute the Frobenius classes for all isomorphism classes
of elliptic curves over~$\F_{17}$ with~$16$ points. The characteristic
polynomial is then~$x^2 + 2x + 17$, with discriminant~$D = 64 = 4^2(-4)$;
the three possible classes over~$ℤ_2$ are~
\[ Φ_0 = \smat{9 & -20 \\ 4 & -7},
Φ_1 = \smat{9 & -40 \\ 2 & -7} \quad\text{and}\quad
Φ_2 = \smat{9 & -80 \\ 1 & -7}.\]
These classes are distinct over~$ℤ/4ℤ$.
\begin{itemize}
\item The curve $E_{9}: y^2 = x^3 + 10x + 6$ with~$j = 9$ has
the three 2-torsion points~$P_{1,1} = (-3, 0)$, $P_{1, 2} = (2, 0)$
and~$P_{1,1} + P_{1,2} = (1, 0)$. This implies that $φ_{E_9}$~is trivial
modulo~$2$.

Let~$P_{2,i} = (x_i, y_i)$ be a point
such that~$2 P_{2,i} = P_{1,i}$: then the coordinate~$x_i$ satisfy the
equations~$x_1^4 + 12 x_1^3 + 14 x_1^2 + 4 x_1 + 2 = 0$ and~$x_2^4 + 60
x_2^3 + 14 x_2^2 + 8 x_2 + 1 = 0$. It is readily checked that no such
point exist with~$(x_i^{17}, y_i^{17}) = λ_i (x_i, y_i)$ with~$λ_i ∈
\acco {1, 3}$. Therefore, $φ_{E_{9}}$ is not diagonalizable modulo~$4$.
This proves that $φ_{E_9}$~is conjugated to~$Φ_1$ in~$GL_2(ℤ_2)$.

\item The curve~$E_{11}: y^2=x^3+x$ has three 2-torsion points~$P_{1,1} =
(0,0)$, $P_{1,2} = (-4, 0)$ and~$(4, 0)$. As before, the equations
for~$x_i$ are $f_1 = x_1^4 - 2 x_1^2 + 1 = 0$ and~$f_2 = x_2^4 - x_2^3 -
2 x_2^2 - x_2 + 1 = 0$. The polynomial~$x^2-1$ divides both~$f_1$
and~$x_1^{17} - x_1$; likewise, $x^2+8x+1$ divides~$f_2$ and~$x_2^{17} -
x_2$. Therefore, $φ_{E_{11}}$~is diagonalizable modulo~$4$ and hence
conjugated to~$Φ_0$ in~$GL_2(ℤ_2)$.

\item The curves~$E_{14}: y^2 = x^3+3x+2$ and~$E_{16}: y^2 = x^3 + 8x -
2$ both have only one $2$-torsion point. Therefore, both Frobenius
endomorphisms are conjugated to~$Φ_2$ in~$GL_2(ℤ_2)$.
\end{itemize}

\end{document}
