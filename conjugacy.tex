\documentclass{article}
\usepackage[T1]{fontenc}
\usepackage[utf8]{inputenc}
\usepackage{lmodern}
\usepackage[english]{babel}
\usepackage{math}
\usepackage{unicode}
\usepackage[margin=15mm]{geometry}

\let\fr\mathfrak
\DeclareMathOperator\Cl{Cl}
\def\smat{\small\mat}

\begin{document}
\title{Conjugacy classes of square $2$-adic matrices of size~$2$}

\begin{prop}
There exists group isomorphisms
\begin{align*}
ℤ_2^{×} &≃ (-1)^{ℤ/2ℤ} × (-3)^{ℤ_2};\\
(ℤ_2/2^m ℤ_2)^{×} &≃ \begin{cases} 1, & m = 1;\\
(-1)^{ℤ/2ℤ} × (-3)^{ℤ/2^{m-2}ℤ},& m ≥ 2.\\\end{cases}
\end{align*}
\end{prop}

\begin{prop}
Let~$ℓ$ be a prime number and $f$~be an irreducible polynomial of
degree~two over~$ℤ_{ℓ}$. The set of all conjugacy classes over~$ℤ_{ℓ}$ of
matrices with characteristic polynomial~$f$ is in bijection with the set
of fractional ideal classes of the ring~$ℤ_{ℓ}[α] / (f(α))$.
\end{prop}

\begin{prop}
Let~$K$ be a quadratic extension of~$ℚ_2$.
\begin{enumerate}
\item Let~$θ ∈ K$ such that~$\ro O_K = ℤ_2[θ]$. Then the orders of~$ℚ_2$
are exactly the rings~$\ro O_n = ℤ_2[2^n θ]$, $n ≥ 0$.
\item The fractional ideal classes of~$\ro O_n$ are the classes of~$\ro
O_m$ for~$0 ≤ m ≤ n$.
\end{enumerate}
\end{prop}

\begin{proof}
A fractional ideal of~$K$ is a $ℤ_2$-lattice~$\fr a ⊂ K$ that is stable
by multiplication by some~$2^n θ$. Moreover, equivalence of fractional
ideals corresponds to equivalence of lattices by multiplication by an
element of~$K^{×}$.

Each lattice is then equivalent to a lattice with basis~$ℤ_2 + τ ℤ_2$
with~$τ ∈ \ro O_K ∖ ℤ_2$. By replacing~$τ$ with~$a τ + b$ with~$a ∈ ℤ_2^{×}$
and~$b ∈ ℤ_2$, we may assume that~$τ = 2^m ω$ for some~$m ≥ 0$,
\emph{i.m.} that $ℤ_2 + τ ℤ_2 = \ro O_m$. The lattice $\ro O_m$~is a $\ro
O_n$-fractional ideal when~$n ≥ m$; conversely, this proves that the
fractional ideals of~$\ro O_n$ are equivalent to the~$\ro O_m$ with~$m ≤
n$.
\end{proof}

\begin{prop}\label{prop:conj-2x2-Z2}
Let~$f(x) = x^2 + px + q$ be a separable polynomial of degree~two
over~$ℤ_2$, and write~$D = p^2 - 4q = d r^2$ with~$r ∈ ℤ_2$ and~$d ∈
\acco {1, -3, -4, 12, ±8, ±24}$. Then there exists exactly~$v_2(r) + 1$
conjugacy classes over~$ℤ_2$ of matrices with characteristic
polynomial~$f$, which are represented by
\begin{equation}\label{eq:conj}
\mat { \frac{-p-rd}{2} & 2^m r \frac{d-d^2}{4} \\
  2^{-m} r & \frac{-p+rd}{2} }, \quad
\text{for $m = 0, …, v_2(r)$.}
\end{equation}
\end{prop}


\begin{prop}\label{prop:conj-2x2-modulo}
For any integer~$s ≥ 0$, the conjugacy classes of matrices with
characteristic polynomial~$f(x) = x^2 + px + q$ over~$ℤ/2^sℤ$ are the
$\min (s, v_2(r)) + 1$ classes given by the formula~\eqref{eq:conj}
with~$m ≥ \max (0, v_2(r) - s)$.
\end{prop}

\begin{prop}
Let~$k$ be a finite field and $E$~be an elliptic
curve over~$k$. For any prime number~$ℓ$ different from the
characteristic of~$k$, the conjugacy class of the Frobenius
endomorphism~$φ_E$ over~$ℤ_{ℓ}$ corresponds to the $ℓ$-adic valuation of
the conductor of the endomorphism ring of~$E$.
\end{prop}


\begin{proof}
By Tate's theorem, the endomorphism ring~$\End_k E$ is isomorphic to the
ring of Galois-equivariant endomorphisms of the Tate module~$T_{ℓ}(E)$.
Let~$\fr a$ be a fractional ideal representing~$φ_E$; then $\End_k E$~is
isomorphic to~$\End_{ℤ_{ℓ}[φ_E]} (\fr a)$. Let~$\ro O$ be the unique
order of~$ℚ_{ℓ}[φ_E]$ equivalent to~$\fr a$; then we see that $\End_k E =
\fr A$.
\end{proof}
\end{document}
