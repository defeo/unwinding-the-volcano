\documentclass{article}
\usepackage[T1]{fontenc}
\usepackage[utf8]{inputenc}
\usepackage{lmodern}
\usepackage[english]{babel}
\usepackage{math}
\usepackage{unicode}
\usepackage[margin=15mm]{geometry}

\let\fr\mathfrak
\DeclareMathOperator\Cl{Cl}
\def\smat{\small\mat}

\begin{document}
\title{Conjugacy classes of square $2$-adic matrices of size~$2$}

\begin{prop}
There exists group isomorphisms
\begin{align*}
ℤ_2^{×} &≃ (-1)^{ℤ/2ℤ} × (-3)^{ℤ_2};\\
(ℤ_2/2^m ℤ_2)^{×} &≃ \begin{cases} 1, & m = 1;\\
(-1)^{ℤ/2ℤ} × (-3)^{ℤ/2^{m-2}ℤ},& m ≥ 2.\\\end{cases}
\end{align*}
\end{prop}

\begin{prop}
Let~$f$ be an irreducible polynomial of degree~two over~$ℤ_2$.
The set of all conjugacy classes over~$ℤ_2$ of matrices with
characteristic polynomial~$f$ is in bijection with the set of fractional
ideal classes of the ring~$ℤ_2[α] / (f(α))$.
\end{prop}

\begin{prop}
Let~$K$ be a quadratic extension of~$ℚ_2$.
\begin{enumerate}
\item Let~$θ ∈ K$ such that~$\ro O_K = ℤ_2[θ]$. Then the orders of~$ℚ_2$
are exactly the rings~$\ro O_n = ℤ_2[2^n θ]$, $n ≥ 0$.
\item The fractional ideal classes of~$\ro O_n$ are the classes of~$\ro
O_m$ for~$0 ≤ m ≤ n$.
\end{enumerate}
\end{prop}

\begin{proof}
A fractional ideal of~$K$ is a $ℤ_2$-lattice~$\fr a ⊂ K$ that is stable
by multiplication by some~$2^n θ$. Moreover, equivalence of fractional
ideals corresponds to equivalence of lattices by multiplication by an
element of~$K^{×}$.

Each lattice is then equivalent to a lattice with basis~$ℤ_2 + τ ℤ_2$
with~$τ ∈ \ro O_K ∖ ℤ_2$. By replacing~$τ$ with~$a τ + b$ with~$a ∈ ℤ_2^{×}$
and~$b ∈ ℤ_2$, we may assume that~$τ = 2^m ω$ for some~$m ≥ 0$,
\emph{i.m.} that $ℤ_2 + τ ℤ_2 = \ro O_m$. The lattice $\ro O_m$~is a $\ro
O_n$-fractional ideal when~$n ≥ m$; conversely, this proves that the
fractional ideals of~$\ro O_n$ are equivalent to the~$\ro O_m$ with~$m ≤
n$.
\end{proof}

\begin{prop}
Let~$f(x) = x^2 + px + q$ be a polynomial of degree~two over~$ℤ_2$, and
write~$D = p^2 - 4q = d r^2$ with~$r ∈ ℤ_2$ and~$d ∈ \acco {-1, ±3, ±2,
±6}$. The conjugacy classes over~$ℤ_2$ of matrices with characteristic
polynomial~$f$ are:
\begin{equation}
\mat { \frac{-p-rd}{2} & 2^m r \frac{d-d^2}{4} \\
  2^{-m} r & \frac{-p+rd}{2} }, \quad
\text{for $m = 0, …, v_2(r)$.}
\end{equation}
% \begin{enumerate}
% \item if $d = 1$:
% \begin{equation}
% \mat{\frac{-p+r}{2} & 2^m \\ 0 & \frac{-p-r}{2}}, 0 ≤ m ≤ v_2(r);
% \end{equation}
% \item if $d = -3$:
% \begin{equation}
% \frac{1}{2} \mat{-p-r & 2^m r \frac{d-1}{2}\\ 2^{-m+1} r & -p+r},
% 0 ≤ m ≤ v_2(r);
% % \mat{\frac{-p-r}{2} & r \pa{\frac{d-1}{4}} \\ r & \frac{-p+r}{2}}; \quad
% % \mat{-p/2 & 2^{m-1} rd \\ 2^{-m-1}r & -p/2}, 0 ≤ m ≤ v_2(r) - 1;
% \end{equation}
% \item if $d \not≡ 1 \pmod{4}$: 
% \begin{equation}
% \frac{1}{2} \mat{-p & 2^{m} rd \\ 2^{-m}r & -p}, 0 ≤ m ≤ v_2(r).
% \end{equation}
% \end{enumerate}
\end{prop}

Computing the index~$m$ for a matrix~$\smat{a_1&a_2\\a_3&a_4}$: we
consider~$ℚ_2^2$ as a $K$-vector space via~$e = \smat{1\\0}$, $α·e =
\smat{a_1\\a_3}$ and the lattice~$\fr a = ℤ_2^2 = (1, \frac{α-a_1}{a_3})
e$.

Write~$D = r^2 d$. If $d \not ≡ 1 \pmod{4}$, then $\frac{α-a_1}{a_3} =
\frac{a_4-a_1}{2 a_3} + \frac{r}{2 a_3} √{d}$. Then $m = v_2 (r/2a_3)$.
If $d ≡ 5 \pmod{8}$, then $\frac{α-a_1}{a_3} = \frac{a_1 - a_4 + r a_3}{2
a_3} - \frac{r}{a_3} ω$; therefore $m = v_2(r/a_3)$.

\end{document}
